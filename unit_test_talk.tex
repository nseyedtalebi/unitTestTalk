\documentclass{beamer}

\usepackage{default}
\usepackage{csquotes}[]


\begin{document}
\author{Nima Seyedtalebi}
\title{Is Unit Testing Worthwhile?}
%\logo{}
\institute{University of Kentucky}
\date{November 7, 2018}
%\subject{}
%\setbeamercovered{transparent}
%\setbeamertemplate{navigation symbols}{}

\begin{frame}[plain]
\maketitle
\end{frame}

\begin{frame}
\frametitle{Background}
\begin{itemize}
	\item The IEEE Software Engineering Body of Knowledge (SWEBOK) provides a concise definition of software testing:
	
		\blockcquote{SWEBOK}{Software testing consists of the \textit{dynamic} verification that a program provides \textit{expected} behaviors on a \textit{finite} set of test cases, suitably \textit{selected} from the usually infinite execution}
 	\item Key points:
	\begin{itemize}
 		\item Dynamic: Input and source code are not always enough to determine behavior
		\item Expected: We must be able to define expected behavior to test for it
 		\item Finite: The set of possible test cases is practically infinite, so we must choose a finite subset
 		\item Selected: Test cases can vary in usefulness considerably, so the choice is important
	\end{itemize}
\end{itemize}
\end{frame}

\begin{frame}
\frametitle{Different Kinds of Testing}
\begin{itemize}
	\item Testing can be classified by target or objective
	\item Classifying by target gives three levels:
	\begin{itemize}
		\item Unit Testing: Small pieces of software testable in isolation
		\item Integration Testing: Interactions between software components
		\item System Testing: An entire system
	\end{itemize}
	\item Classifications by objective: 
	\begin{itemize}
		\item Regression testing
		\item Acceptance testing
		\item Security testing
		\item Performance testing
		\item Stress testing
	\end{itemize}
\end{itemize}
\end{frame}

\begin{frame}
\frametitle{What is Unit Testing?}
\begin{itemize}
	\item From the SWEBOK:\blockcquote{SWEBOK}{Unit testing verifies the functioning in isolation of software elements that are separately testable.}
	\begin{itemize}
 		\item What constitutes a unit? It depends on context
 		\item Developers may have differing ideas about what constitutes a unit
	\end{itemize}
	\item Usually performed by the developer of the unit or someone with programming skills and access to the source code
	\item Surveys suggest unit testing is an important testing method that sees widespread use
	\item Unit testing is sometimes conflated with other kinds of testing
	\begin{itemize}
		\item E.g. a "unit test" that relies on a database connection is not a unit test under the definition given
	\end{itemize}
\end{itemize}
\end{frame}

\begin{frame}
\frametitle{Challenges with Unit Testing}
\begin{itemize}
	\item The test oracle problem: We must know beforehand what the code is supposed to do
	\item Exhaustive testing is impractical at best and impossible at worst
	\item Consider a program that takes a Unicode string as an argument and writes it to STDOUT:
	\begin{itemize}
		\item The Unicode 11.0 standard contains 137,374 different characters\cite{unicodestd},so $137374^n$ permutations of length $n$
		\item Such a program depends on the behavior of STDOUT. Truly exhaustive testing would have to account for this
	\end{itemize}
	\item Some tests are more useful than others. How do we choose the best set of tests?
	\item How do we know if we have enough tests?
	\item How do we know if unit testing is effective?
\end{itemize}
\end{frame}

\begin{frame}
\frametitle{Common Test Techniques}
\begin{itemize}
	\item Ad-hoc: Choose test inputs based on intuition and experience
	\item Boundary-value Analysis: Choose inputs close to boundaries in the input domain e.g. largest and smallest possible values for numerical datatypes
	\item Control Flow Analysis: Choose tests that follow the different possible paths of execution in the code
	\begin{itemize}
		\item Statement coverage, Branch Coverage, and Decision/Condition coverage are different kinds of coverage
		\item Used as a measure of test sufficiency as well
	\end{itemize}
\end{itemize}
\end{frame}

\begin{frame}
\frametitle{Software Testing Metrics}
\begin{itemize}
	\item Statement coverage. Use example from 1978 paper about why two cases is not enough to fully test even a simple conditional
\end{itemize}
\end{frame}



\begin{frame}
\frametitle{Tool Support}
\begin{itemize}
	\item Testing frameworks
	\item JUnit, Mockito, PowerMock in particular
	\item Continuous Integration
\end{itemize}
\end{frame}

\begin{frame}
\frametitle{Arguments For}
\begin{itemize}
	\item 
\end{itemize}
\end{frame}
\end{document}

\begin{frame}
\frametitle{Arguments Against}
\begin{itemize}
	\item 
\end{itemize}
\end{frame}

\begin{frame}
\frametitle{Limitations of Unit Testing}
\begin{itemize}
	\item 
\end{itemize}
\end{frame}

\begin{frame}
\frametitle{How Can We Assess Testing Behaviors and Outcomes Empirically?}
\begin{itemize}
	\item How do we measure software quality? What does that even mean?
	\item Callback to metrics slide
\end{itemize}
\end{frame}

\begin{frame}
\frametitle{What Do the Available Data Suggest?}
\begin{itemize}
	\item 
\end{itemize}
\end{frame}

\begin{frame}
\frametitle{What About Test-Driven Development (TDD)?}
\begin{itemize}
	\item What is it?
	\item What are the benefits?
	\item How does TDD related to unit testing practice?
	\item How does TDD affect software quality?
\end{itemize}
\end{frame}

\begin{frame}
\frametitle{Pitfalls to Avoid}
\begin{itemize}
	\item 
\end{itemize}
\end{frame}

\begin{frame}
\frametitle{There's Work to be Done!}
\begin{itemize}
	\item 
\end{itemize}
\end{frame}

\begin{frame}
\frametitle{Conclusions}
\begin{itemize}
	\item 
\end{itemize}
\end{frame}

\begin{frame}
\frametitle{References}
	\bibliographystyle{plain}
	\bibliography{unit_test_talk}
\end{frame}

\end{document}
